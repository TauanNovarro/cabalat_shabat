

\chapter*{Nota da organizadora\footnote{ 
	A publicação surgiu a partir da realização de uma cerimônia de 
	\emph{cabalat shabat}, proposta pelo Instituto Brasil"-Israel (\versal{IBI}), na qual participantes judeus e não"-judeus puderam encontrar uma forma de expressão, contemplação e exercício de significação própria. Como produtora do \versal{IBI}, me incumbi da tarefa de reunir nesta coletânea bênçãos tradicionalmente recitadas no recebimento do \emph{shabat}, que materializam a preservação da tradição judaica ao passo que a revisitamos.
	Este trabalho só foi possível graças à amizade da Suzana Salama, do Gabriel Neistein e da Marília Neustein.} }
\addcontentsline{toc}{chapter}{Nota da organizadora, \emph{por Fabiana Gampel Grinberg}}


\begin{flushright}
\emph{Fabiana Gampel Grinberg}
\end{flushright}

%\begin{multicols}{2}[]

Este pequeno guia bilíngue, de tradução livre e interpretativa,\footnote{A tradução não literal dos textos é proposital e está alinhada à visão humanista, progressista e cultural do judaísmo. Ver nota p.~\pageref{ref01}.}
 se propõe a acolher um ponto de vista amplo da espiritualidade que não se atenha ao pé da letra dos registros religiosos. Nossa proposta para este \emph{cabalat shabat} e  para a mensagem do texto de abertura dessa \emph{leket} ( {לקט}\,, ``coleção'', ``antologia'') diz respeito a aprender e compartilhar o que é significativo para cada um como indivíduo, dentre as muitas maneiras de celebração do judaísmo e de sua herança. {Abençoado seja o Espírito do Universo, Fonte da Eternidade, que nos dá a habilidade de questionar}.
\emph{Todá rabá!}

%\end{multicols}

\chapter*{Introdução ao \emph{cabalat shabat}}
\addcontentsline{toc}{chapter}{Introdução ao \emph{cabalat shabat}, \emph{por Gabriel Neistein}}


\begin{flushright}
\emph{Gabriel Neistein}
\end{flushright}

%טאלאבק הלאבק
%\begin{multicols}{2}

A palavra hebraica קבאלת\,, 
\emph{cabalat}, vem de \emph{cabalá}, ``recebimento''. A
origem da cerimônia de \emph{cabalat shabat} nasce na mística de Isaac Luria (1534--1572),
cabalista de Tzfat (ou Safed),\footnote{O cabalismo surgiu originalmente no sul da França, mas alcançou pleno desenvolvimento na Espanha do século \versal{XIII}. Ganhou entretanto novo significado a partir do século \versal{XVI} ao fornecer uma espécie de resposta à questão do exílio dos judeus ibéricos em 1492. No século \versal{XVII} foi vinculado ao movimento messiânico de Sabatai Tzvi, considerado herético à época, e que mesmo em colapso provocou a omissão da cabala dos círculos judaicos oficiais. Isaac Luria, o ``Ari'', foi o maior e mais conhecido nome dentre os cabalistas, originando a partir de seus ensinamentos a Escola Luriânica. Embora nascido em Jerusalém, estabeleceu"-se na cidade de Safed --- que se tornou o centro cabalista mundial --- por volta de 1569.} e simboliza o recebimento
da \emph{schechiná}, literalmente, ``assentamento'', ``habitação'' ou ``moradia'', que 
em sentido alegórico significa ``habitação'' ou ``presença de Deus''. 

Apresentada sob a figura de uma noiva no poema \emph{Lechá Dodi},\footnote{Literalmente ``Vem, meu amado''. Alguns ritos judaicos foram influenciados pela cabala, principalmente os que adotaram as bodas sagradas em suas cerimônias --- representadas pela união entre os aspectos femininos e masculinos ou Deus, o rei e a rainha, ou mais precisamente a \emph{Shechiná} e a mística \emph{Ecclesia} de Israel. Popularmente foram entendidas como uma união entre Deus e Israel, o que, para os cabalistas, representa apenas o aspecto externo da interioridade secreta de Deus.} cantiga do 
serviço religioso do \textit{cabalat shabat}, trata-se de uma
presença serena que conduz à paz nas sinagogas e espaços privados.

Em caráter introdutório à noção de \emph{shabat}, reúno a seguir duas
passagens sobre o termo. A primeira diz respeito a uma pequena história chassídica 
recolhida pelo filósofo e escritor  
Martin Buber (1878--1975), em suas viagens pelos \emph{shtetls}:\footnote{Povoações ou bairros 
judaicos da Europa Oriental.}

\begin{quote}
Semana após semana, com a chegada do \emph{shabat}, os irmãos Rabi Zússia
e o Rabi Elimelech eram tomados de grande sentimento de santidade. Uma
vez disse o Rabi Elimelech ao Rabi Zússia:

--- Irmão, às vezes tenho medo de que meu sentimento de santidade
no \emph{shabat} não seja verdadeiro, que seja apenas imaginação.

--- Irmão -- disse Zússia --- eu também tenho, às vezes, este medo.

--- O que vamos fazer? --- perguntou Elimelech.

Zússia respondeu:

--- Vamos cada um de nós, num dia qualquer da semana, preparar uma
refeição, exatamente igual ao jantar de \emph{shabat}, sentar"-nos entre os
\emph{chassidim} [tementes a Deus] e dizer palavras dos ensinamentos.

Assim fizeram: prepararam uma completa refeição de \emph{shabat},
vestiram roupas limpas, puseram os gorros de pele, comeram no meio dos
\emph{chassidim} e disseram palavras dos ensinamentos. Então desceu sobre eles
um imenso sentimento de santidade, como se fosse \emph{shabat}.\footnote{\emph{Histórias do Rabi}, Martin Buber (tradução M. Arnsdorff, T. Belinky, J. Guinsburg, R. Mautner, R. Schivartche, R. Simis), Editora Perspectiva, 2000.}
\end{quote}

O pequeno texto pode ser entendido como uma busca pela essência
do \emph{shabat} e a diferença entre esses dias e o resto da semana. No judaísmo é o anoitecer, amanhecer ou entardecer que trazem a santidade. E a espiritualidade do sétimo dia vai de encontro às formas significantes do tempo: ``o \emph{shabat} não é uma data, mas uma atmosfera''.\footnote{\emph{O schabat: seu significado para o homem moderno}, Abraham Joshua Heschel (tradução Fany Kon \& Jacob Guinsburg), Editora Perspectiva, 2000, p.~32.}

Em seguida, transcrevo uma passagem do livro \emph{O schabat}, do rabino e teólogo 
Abraham J. Heschel (1907--1972):

\begin{quote}
Este exato momento pertence a todos os homens vivos, tal como me
pertence. Nós partilhamos o tempo, nós possuímos o espaço. Pelo fato de
eu possuir o espaço, sou um rival de todos os outros seres; através da
minha existência no tempo, eu sou um contemporâneo de todos os outros
vivos.

O significado do \emph{shabat} é, antes, o de celebrar o tempo, e não o
espaço. Seis dias da semana vivemos sob a tirania das coisas do espaço;
no \emph{shabat} tentamos nos tornar harmônicos com a santidade no tempo. É um
dia em que somos chamados a partilhar no que é eterno no tempo, para
fugir dos resultados da criação, para os mistérios da criação; do mundo
da criação para a criação do mundo.\footnote{Idem, p.~18.}
\end{quote}

A partir dos textos e desenvolvimentos expostos, podemos entender
o recebimento do \emph{shabat} e sua vivência durante a noite de sexta"-feira e o dia de sábado como uma modificação de consciência. É um tempo festivo, meditativo, prazeroso e fora da civilização técnica --- conceito apresentado no mesmo livro de Heschel, exposto acima e em notas anteriores, que se refere à sociedade do produto e do ganho e ao poder do homem.

Guardar o \emph{shabat} é, portanto, guardar o tempo. Mas o mais importante é lembrar que ``estamos dentro do \emph{shabat} mais do que o \emph{shabat} está dentro de nós''.\footnote{Idem, p.~32.}

%\end{multicols}

\chapter*{}
\addcontentsline{toc}{part}{Cabalat shabat}
\begin{center}
\begin{vplace}[0.3]
\Large
Cabalat shabat
\end{vplace}
\end{center}
\thispagestyle{empty}


\movetoevenpage
\raggedleft

\vspace*{1cm}

\addcontentsline{toc}{chapter}{Bênção das velas}
\textsc{bênção das velas}\\[15pt]

אַדָנָי אַתָּה בָּרוּך

הָעוֹלָם מֶלֶך אֱלהֵינוּ

בְּמִצְוֹתָיו קִדְשָנוּ אַשֶׁר

שַבָּת שֶל נֵר לְהַדְלִיק וְצִוָנוּ‏\footnote{Baruch atá Adonai,/ Eloheinu melech haolam,/ asher kidshanu bemitzvotáv,/ vetzivanu lehadlic ner shel shabat.}

\movetooddpage
\raggedright

\vspace*{1cm}

\textsc{}\\[15pt]

Abençoada seja

a \emph{Eterna Fonte de Luz}, 

que nos ilumina como a chama \label{ref01}

das velas do \emph{shabat}.\footnote{Literalmente ``Abençoado seja o senhor, rei do mundo''. A expressão ``Eterna Fonte de Luz'' é uma tradução alegórica, que indica uma imagem menos centralizadora ou hierarquizada de D'us. Muitos termos similares aparecerão ao longo dos poemas rituais --- atentando também para o fato de que o nome de D'us no judaísmo não é pronunciado e por isso é possível representá"-lo de diversas maneiras. Nesse caso específico, ``Eterna Fonte de Luz'' remete à simbologia das duas velas que são acesas no anoitecer da sexta"-feira, e que correspondem às expressões שמור\,, \emph{shamor}, ``lembrar'' (``Recordar o dia de \emph{shabat} para santificá"-lo''. Êxodo \versal{XX}:~8) e זכור\,, \emph{zachor}, ``guardar'' (``Guardar o dia do \emph{shabat} para santificá"-lo''. Deuteronômio \versal{V}:~12), mencionados nos Dez Mandamentos.}


\movetoevenpage
\raggedleft

\vspace*{1cm}

\addcontentsline{toc}{chapter}{Iedid Nefesh}
\textsc{iedid nefesh}\\[15pt]

הָרַחְמָן אָב נֶפֶשׁ, יְדִיד 

רְצונָךְ אֶל עַבְדָךְ מְשךְ 

אַיָל כְמו עַבְדָךְ יָרוּץ 

הֲדָרָךְ מוּל יִשְתַחֲוֶה 

יְדִידוּתָךְ לוֹ יֶעְרַב כִּי 

טָעַם וְכָל צוּף מִנּפֶת\footnote{Iedid nefesh av harachaman,/ meshoch avdecha el retzonecha,/ iarutz avdecha kemo aial,/ ishtachavê el mul hadarecha./ Ki ierav lo iedidotecha,/ minofet tzuf vechol taam.}\\[10pt]

הָעולָם זִיו נָאֶה, הָדוּר,

אַהֲבָתָךְ חולַת נַפְשִי 

לָהּ נָא רְפָא נָא, אֵל אָנָא 

זִיוָךְ נעַם לָהּ בְּהַרְאות

וְתִתְרַפֵּא תִתְחֵזֵּק אָז

עולָם שִׂמְחַת לָךְ וְהָיְתָה\footnote{Hadur naê ziv haolam,/ nafshi cholat ahavatecha,/ Ana El na, refá na la,/ beharot la noam zivechá./ Az titchazek vetitrapê,/ vehaietah la simchat olam.}\\[10pt]


\movetooddpage
\raggedright

\vspace*{1cm}

\textbf{}\\[15pt]

\emph{Amado da minha alma},

me chame em sua direção,

correrei como um cervo,

para admirar toda a sua majestade.

Receber seu afeto é para mim

mais doce que todo o mel.\\[10pt]

\emph{Fonte de toda a glória que há no mundo},\footnote{Ver página \pageref{ref01}.}

minha alma arde de amor por \emph{Você}.

Por favor, cure-a,

e mostre-me toda a beleza de seu esplendor.

Então serei forte,

e será completa a minha alegria.\\[10pt]

\movetoevenpage
\raggedleft

\vspace*{1cm}

רַחֲמֶיךָ נָא יֶהֱמוּ וָתִיק \label{ref02}

אוֹהֲבָךְ בֵּן עַל נָא וְחוּס 

נִכְסַפְתִּי נִכְסוֹף כַּמֶּה זֶה כִּי 

עֻזָךְ בְּתִפְאֶרֶת לִרְאות 

לִבִי מַחְמָד אֵלִי, אָנָא 

תִּתְעַלָם וְאַל נָא, חוּסה\footnote{Vatik iehemu na rachamecha,/
vechusá na al ben ahuvecha,/ ki ze cama nichsof nichsafti,/ lirot betiferet uzechá,/
ana eli machmad libi,/ vechusá na veal titalam.}\\[10pt]

חָבִיב וּפְרשׂ, נָא הִגָלֵה

שְלומֶךְ סֻכַת אֶת עָלַי 

מִכְּבוֹדָךְ אֶרֶץ תָּאִיר 

בָךְ וְנִשְׂמְחָה נָגִילָה 

מועֵד בָא כִּי אָהוּב, מַהֵר

עולָם כִּימֵי וְחָנֵנִי\footnote{
Higaleh na ufrós chavivi alai,/
et sucat shelomecha/ Tair eretz mikvodecha,/ nagila venismechah bach./ Maher ahuv ki va moed,/
vechoneinu kimei olam.}

\movetooddpage
\raggedright

\vspace*{1cm}

\emph{Fonte da Eternidade},\footnote{Ver página \pageref{ref01}.}

me acolha como uma criança,

pois tanta é a minha vontade

de admirar todo o seu encanto.

Este é o desejo do meu coração,

não se esconda.\\[10pt]

Revele-se, \emph{Fonte Eterna do Amor},\footnote{Ver página \pageref{ref01}.}

e estenda sobre mim um manto de paz,

e ilumine todo o mundo

para que todos possam brilhar de alegria.

Depressa, \emph{meu amado}, esta é a hora,

se aproxime e me abrace pela eternidade.

\movetoevenpage
\raggedleft
\addcontentsline{toc}{chapter}{Lechá Dodi}

\vspace*{1cm}

\textsc{lechá dodi}\\[15pt]

כַּלָּה לִקְרַאת דוֹדִי לְכָה

נְקַבְּלָה שַׁבָּת פְּנֵי\footnote{Lechá dodi licrat calá,/ penei shabat necabelá.}\\[10pt]

אֶחָד בְּדִבּוּר וְזָכוֹר שָׁמוֹר

הַמְּיֻחָד אֵל הִשְׁמִיעָנוּ

אֶחָד וּשְׁמוֹ אֶחָד ה' 

וְלִתְהִלָּה וּלְתִפְאֶרֶת לְשֵׁם\footnote{Shamor vezachor bedibur echad,/ Hishmianu el hameiuchad./ Adonai echad ushemó echad,/ Leshem uletiferet veletehila.}\\[10pt]

וְנֵלְכָה לְכוּ שַׁבָּת לִקְרַאת

הַבְּרָכָה מְקוֹר הִיא כִּי

נְסוּכָה מִקֶּדֶם מֵרֹאשׁ 

תְּחִלָּה בְּמַחֲשָׁבָה מַעֲשֶּׂה סוֹף\footnote{Licrat shabat lechu venelchá/ Ki hi mecor haberachá,/ merosh mikedem nesuchá,/ sof maassê bemachshavá tehilá.}\\[10pt]

מִקְדַּשׁ מֶֽלֶךְ עִיר מְלוּכָה

קֽוּמִי צְאִי מִתּוֹךְ הַהֲפֵכָה

רַב לָךְ שֶֽׁבֶת בְּעֵֽמֶק הַבָּכָא

וְהוּא יַחֲמוֹל עָלַֽיִךְ חֶמְלָה\footnote{Inserir transliteração.}\\[10pt]


\movetooddpage
\raggedright

\vspace*{1cm}

\textsc{}\\[15pt]

Vem, \emph{meu amado}, encontrar a noiva.

Venha receber a presença do \emph{shabat}.\\[10pt]

\emph{Guardar e Lembrar} são

duas palavras em uma só expressão,

o \emph{Eterno} nos revela que é \emph{Um},

que \emph{Seu nome} é \emph{Único} e por isso cantamos.\\[10pt]

Vamos receber o \emph{shabat},

que é a origem de todas as bênçãos,

desde os tempos mais antigos,

a última ação é também o primeiro pensamento.\\[10pt]

Templo santo, cidade real,

Emerge das ruínas em meio à agitação.

Já não habita mais o Vale de Lágrimas,

A era da compaixão chegou.\\[10pt]


\movetoevenpage
\raggedleft

\vspace*{1cm}

\textsc{}\\[15pt]


הִתְנַעֲרִי מֵעָפָר קוּמִי

לִבְשִׁי בִּגְדֵי תִפְאַרְתֵּךְ עַמִּי

עַל יַד בֶּן יִשַׁי בֵּית הַלַּחְמִי

קָרְבָה אֶל נַפְשִׁי גְאלָּהּ\footnote{Inserir transliteração.}\\[10pt]

הִתְעוֹרְרִי הִתְעוֹרְרִי

אוֹרִי קוּמִי אורֵךְ בָא כִּי

דַּבֵּרִי שִׁיר עוּרִי עוּרִי

נִגְלָּה עָלַיִךְ ה' כְּבוֹד\footnote{Hitoreri, hitoreri,/ Ki va orech cumi ori,/ Uri uri shir daberi,/
Kvod Adonai alaich niglá.}\\[10pt]

בַּעְלָהּ עֲטֶרֶת בְשָׁלוֹם בֹּאִי

וּבְצָהֳלָה בְּשִּׂמְחָה גַּם 

סְגֻלָּה עַם אֱמוּנֵי תּוֹךְ 

כַלָּה בּוֹאִי כַלָּה בּוֹאִי\footnote{
Boi veshalom ateret baalá,/ Gam besimchá uvetsarlá,/
Toch emunei am segulá,/ Boi calá, boi calá.}

\movetooddpage
\raggedright

\vspace*{1cm}

\textsc{}\\[15pt]


O pó que se levanta das cinzas é sacudido,

E são vestidas roupas esplêndidas.

Através do filho de Ishai, de Belém,

A liberdade se aproxima.\\[10pt]

Desperta, acorda,

pois resplandece o brilho,

levanta e entoa a melodia,

os raios da \emph{Luz} nos aquecem.\\[10pt]

Venha em paz, noiva,

com música e alegria.

Te recebemos com apreço,

venha noiva, venha noiva.

\movetoevenpage
\raggedleft
\addcontentsline{toc}{chapter}{Shalom Aleichem}

\vspace*{1cm}

\textsc{shalom aleichem}\\[15pt]

עֶלְיוֹן מַלְאֲכֵי הַשָרֵת מַלְאֲכֵי עֲלֵיכֶם שָלוֹם

הוּא בָרוּךְ הַקָדוֹשׁ הַמְלָכִים מַלְכֵי מִמֶלֶךְ\footnote{Shalom alechem malachei, hasharet malachei Elion,/
mimelech malchei hamelachim hacadosh Baruch Hu.}\\[10pt]

עֶלְיוֹן מַלְאֲכֵי הַשָּׁלוֹם מַלְאֲכֵי לְשָׁלוֹם בּוֹאֲכֶם

הוּא בָרוּךְ הַקָדוֹשׁ הַמְלָכִים מַלְכֵי מִמֶלֶךְ\footnote{Boachem leshalom malachei, hashalom malachei Elion,/
mimelech malchei hamelachim hacadosh Baruch Hu.}\\[10pt]

עֶלְיוֹן מַלְאָכִי הַשָּׁלוֹם מַלְאֲכֵי לְשָלוֹם בָרְכוּנִי

הוּא בָרוּךְ הַקָדוֹשׁ הַמְלָכִים מַלְכֵי מִמֶלֶךְ\footnote{Barechuni leshalom malachei, hashalom malachei Elion,/
mimelech malchei hamelachim hacadosh Baruch Hu.}\\[10pt] 

עֶלְיוֹן מַלְאָכִי הַשָּׁלוֹם מַלְאֲכֵי לְשָלוֹם צֵאתְכֶם 

הוּא בָרוּךְ הַקָדוֹשׁ הַמְלָכִים מַלְכֵי מִמֶלֶךְ\footnote{Tsetechem leshalom malachei, hasharet malachei Elion,/
mimelech malchei hamelachim hacadosh Baruch Hu.} 


\movetooddpage
\raggedright

\vspace*{1cm}

\textsc{}\\[15pt]

Estejam em paz, anjos protetores, mensageiros do \qb{}infinito,

da suprema \emph{Divindade},  do que é santo, do que é \qb{}abençoado.\\[10pt]

Que venham em paz, os anjos da paz, mensageiros do \qb{}infinito,

da suprema \emph{Divindade}, do que é santo, do que é \qb{}abençoado.\\[10pt]

Abençoem-me com a paz, anjos da paz, mensageiros do \qb{}infinito,

da suprema \emph{Divindade}, do que é santo, do que é \qb{}abençoado.\\[10pt]

Que partam em paz, os anjos protetores, mensageiros do \qb{}infinito,

da suprema \emph{Divindade}, do que é santo, do que é \qb{}abençoado.



\movetoevenpage
\raggedleft

\addcontentsline{toc}{chapter}{Kidush}

\vspace*{1cm}

\textsc{kidush}\\[15pt]

צְבָאָם וְכָל וְהָאָרֶץ הַשָּׁמַיִם וַיְכֻלּוּ 

עָשָׂה אֲשֶׁר מְלַאכְתּו הַשְּׁבִיעִי בַּיּום אֱלהִים וַיְכַל 

עָשָׂה אֲשֶׁר מְלַאכְתּו מִכָּל הַשְּׁבִיעִי בַּיּום וַיִּשְׁבּת 

אתו וַיְקַדֵּשׁ הַשְּׁבִיעִי יום אֶת אֱלהִים וַיְבָרֶךְ 

לַעֲשׂות אֱלהִים בָּרָא אֲשֶׁר מְלַאכְתּו מִכָּל שָׁבַת בו כִּי\footnote{Vaichulu hashamaim vehaaretz vechol tzevaam,/
Vaichal Elohim baiom hashvií melachtó asher asá,/
Vaishbot baiom hashvií micol melachtó asher asá,/
Vaivarech Elohim et iom hashvií vaicadeish oto,/
Ki vo shavat micol melachtó asher bará Elohim laasot.}\\[10pt] 

הַגָּפֶן פְּרִי בּורֵא הָעולָם מֶלֶךְ אֱלהֵינוּ אַדָנָי אַתָּה בָּרוּךְ\footnote{Baruch atá, Adonai Eloheinu, Melech haolam, borê peri hagafen.}\\[10pt] 

הָעולָם מֶלֶךְ אֱלהֵינוּ ה' אַתָּה בָּרוּךְ

הִנְחִילָנוּ וּבְרָצון בְּאַהֲבָה קָדְשׁו וְשַׁבַּת

בָנוּ. וְרָצָה בְּמִצְותָיו קִדְּשָׁנוּ אֲשֶׁר

בְרֵאשִׁית לְמַעֲשֵׂה זִכָּרון

מִצְרָיִם לִיצִיאַת זֵכֶר קדֶשׁ לְמִקְרָאֵי תְּחִלָּה יום הוּא כִּי\footnote{Baruch atá, Adonai Eloheinu, Melech haolam, asher kideshanu bemitzvotav veratza/
vanu, ve shabat codsho beahavá uveratzon hinchilanu, zikaron lemaasê vereishit./
Ki hu iom techila lemikraei codesh, zecher litziat Mitzrayim/ Ki vanu vacharta, veotanu kidashta, micol haamim.}\\[10pt]


\movetooddpage
\raggedright

\vspace*{1cm}

\textsc{}\\[15pt]

Os céus e a terra e todos que vivem lá foram criados,

a obra da \emph{Fonte da Criação} completou-se no sétimo dia,

e no sétimo dia contemplou-se todo o trabalho,

e este dia foi escolhido e abençoado,

pois foi quando houve descanso

de tudo o que havia para ser criado.\\[10pt]

Abençoada seja a \emph{Eterna Fonte da Vida}\footnote{Ver página \pageref{ref01}.} que faz crescer o \qb{}fruto da videira.\\[10pt]

Abençoada seja a \emph{Eterna Fonte de Força},\footnote{Ver página \pageref{ref01}.} que nos permite \qb{}escolher nossas ações.

Recebemos o \emph{shabat} com amor, como herança e como \qb{}memória.

E o escolhemos como a lembrança do refúgio de nossas \qb{}lutas,

como o dia em que conquistamos a liberdade na saída do \qb{}Egito.\\[10pt]


\movetoevenpage
\raggedleft

\vspace*{1cm}


הִנְחַלְתָּנוּ וּבְרָצון בְּאַהֲבָה קָדְשְׁךָ וְשַׁבַּת הָעַמִּים מִכָּל קִדַּשְׁתָּ וְאותָנוּ בָחַרְתָּ בָנוּ כִּי

הַשַּׁבָּת מְקַדֵּשׁ אַדָנָי אַתָּה בָּרוּךְ\footnote{Ve shabat codshechá beahava uveratzon hinchaltanu,/
Baruch atá, Adonai, mecadêsh ha shabat.}

\movetooddpage
\raggedright

\vspace*{1cm}


Abençoada seja a \emph{Eterna Fonte da Vida}\footnote{Ver página \pageref{ref01}.} que cria o amor, a alegria, a música e o prazer, e nos presenteia com o \emph{shabat}.

\movetoevenpage
\raggedleft

\addcontentsline{toc}{chapter}{Benção da chalá}

\vspace*{1cm}

\textsc{benção da chalá}\\[15pt]

אַדָנָי אַתָּה בָּרוּך

הָעוֹלָם מֶלֶך אֱלהֵינוּ 

הַארץ מן לֶחם הָמוֹציא\footnote{Baruch atá Adonai,/ Eloheinu melech haolam,/ hamotzi lechem min haaretz.}

\movetooddpage
\raggedright

\vspace*{1cm}

\textsc{}\\[15pt]

Abençoada seja

a \emph{Eterna Fonte da Vida},\footnote{Ver página \pageref{ref01}.} 

que nos nutre com o alimento da terra.
